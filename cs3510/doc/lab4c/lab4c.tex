\documentclass[]{article}
\usepackage[T1]{fontenc}
\usepackage{lmodern}
\usepackage{amssymb,amsmath}
\usepackage{ifxetex,ifluatex}
\usepackage{fixltx2e} % provides \textsubscript
% use microtype if available
\IfFileExists{microtype.sty}{\usepackage{microtype}}{}
% use upquote if available, for straight quotes in verbatim environments
\IfFileExists{upquote.sty}{\usepackage{upquote}}{}
\ifnum 0\ifxetex 1\fi\ifluatex 1\fi=0 % if pdftex
  \usepackage[utf8]{inputenc}
\else % if luatex or xelatex
  \usepackage{fontspec}
  \ifxetex
    \usepackage{xltxtra,xunicode}
  \fi
  \defaultfontfeatures{Mapping=tex-text,Scale=MatchLowercase}
  \newcommand{\euro}{€}
\fi
\usepackage[margin=0.8in]{geometry}
\ifxetex
  \usepackage[setpagesize=false, % page size defined by xetex
              unicode=false, % unicode breaks when used with xetex
              xetex]{hyperref}
\else
  \usepackage[unicode=true]{hyperref}
\fi
\hypersetup{breaklinks=true,
            bookmarks=true,
            pdfauthor={Sriram V - CS11B058},
            pdftitle={Lab 4 - Part C: Preemptive Multitasking and Inter-Process Communication (IPC)},
            colorlinks=true,
            urlcolor=blue,
            linkcolor=magenta,
            pdfborder={0 0 0}}
\urlstyle{same}  % don't use monospace font for urls
\setlength{\parindent}{0pt}
\setlength{\parskip}{6pt plus 2pt minus 1pt}
\setlength{\emergencystretch}{3em}  % prevent overfull lines
\setcounter{secnumdepth}{0}

\title{Lab 4 - Part C: Preemptive Multitasking and Inter-Process Communication
       (IPC)}
\author{Sriram V - CS11B058}
\date{}

\begin{document}
\maketitle

\subsection{Exercise 13}

\begin{itemize}
\itemsep1pt\parskip0pt\parsep0pt
\item
  Since I had implemented the Challenge question in lab 3, inserting IDT
  entries was trivial. I had to simply increase the number of iterations
  to \texttt{52} for the for loop in \texttt{trap\_init()} in
  \texttt{kern/trap.c}.
\item
  The corresponding \texttt{TRAPHANDLER\_NOEC()} macros are also used in
  \texttt{kern/trapentry.s}.
\item
  To ensure that user environments always run with interrupts enabled,
  we add the \texttt{FL\_IF} flag to the environment's \texttt{EFLAGS}.
\item
  This is done by using
  \texttt{e-\textgreater{}env\_tf.tf\_eflags \textbar{}= FL\_IF;}.
\item
  Now if we run any program that runs for a non-trivial length of time
  (such as \texttt{spin}), we see that the kernel is able to preempt the
  running environment, but prints the trap frames for the hardware
  interrupts as it is not yet handling them.
\end{itemize}

\subsection{Exercise 14}

\begin{itemize}
\itemsep1pt\parskip0pt\parsep0pt
\item
  Here, the \texttt{trap\_dispatch()} function in \texttt{kern/trap.c}
  is modified to handle the timer's hardware interrupt.
\item
  We add in a case for the corresponding \texttt{trapno}, which would be
  handle the case for the IDT entry corresponding to the timer IRQ.
\item
  The mapping from IRQ number to IDT entry is not fixed.
  \texttt{pic\_init} maps IRQs 0-15 to IDT entries \texttt{IRQ\_OFFSET}
  to \texttt{IRQ\_OFFSET+15}.
\item
  Hence, the value we equate \texttt{trapno} to, to check for the timer
  hardware interrupt is \texttt{IRQ\_OFFSET + IRQ\_TIMER}, as this is
  the corresponding IDT entry.
\item
  If the \texttt{trapno} value is indeed equal to this case, we call
  \texttt{lapi\_eoi()} to acknowledge the interrupt, and then call the
  scheduler using \texttt{sched\_yield()}.
\item
  The \texttt{user/spin} test now works. This is because when the child
  environment running \texttt{spin} loops indefinitely, the kernel
  preempts this environment and allocates the system resources to the
  next environment in line.
\item
  Thus we see that the parent environment forks off the child,
  \texttt{sys\_yield()}s to it a couple of times, but in each case
  regains control of the CPU after one time slice, and finally kills the
  child environment and terminates gracefully.
\end{itemize}

\subsection{Exercise 15}

\begin{enumerate}
\def\labelenumi{\arabic{enumi}.}
\itemsep1pt\parskip0pt\parsep0pt
\item
  \textbf{\texttt{sys\_ipc\_recv()}:}

  \begin{itemize}
  \itemsep1pt\parskip0pt\parsep0pt
  \item
    If \texttt{dstva} is less than \texttt{UTOP}, it means that the
    environment is willing to receive a page of data. In such a case, we
    check if \texttt{dstva} must be page-aligned, or else we throw an
    error.
  \item
    If it is page-aligned, \texttt{curenv}'s \texttt{env\_ipc\_dstva} is
    set to \texttt{dstva}, \texttt{env\_ipc\_recving} is set to
    \texttt{1} and \texttt{env\_status} is set to
    \texttt{ENV\_NOT\_RUNNABLE}, before returning a \texttt{0}.
  \item
    This is done to indicate to other environments that this environment
    wants to receive data.
  \end{itemize}
\item
  \textbf{\texttt{sys\_ipc\_try\_send()}:}

  \begin{itemize}
  \itemsep1pt\parskip0pt\parsep0pt
  \item
    Here we first check if the environment associated with the passed
    \texttt{envid} exists using \texttt{envid2env()}, and return an
    error if no such environment exists.
  \item
    \texttt{checkperm} is set to \texttt{0} above, since we don't need
    to check permissions.
  \item
    \texttt{-E\_IPC\_NOT\_RECV} error is thrown if the \texttt{envid} is
    not blocked is \texttt{sys\_ipc\_recv()}, or another environment has
    managed to send first.
  \item
    If \texttt{srcva} is less than \texttt{UTOP}, it means that the
    sender wants to send a page. Also, since the check above has alos
    been passed, it means that \texttt{envid} is willing to receive.
  \item
    If it is not less than \texttt{UTOP}, a page won't be transferred
    and hence \texttt{env\_ipc\_perm} is set to \texttt{0}.
  \item
    However, if it is lesser, we check that \texttt{srcva} is pafe
    aligned and that it has the relevant permssions of being present and
    a user page etc.
  \item
    The page is found by using \texttt{page\_lookup()} at
    \texttt{srcva}, and if the page exists and both the page and its
    page table entry are writable, we proceed forward. Else, we throw an
    error.
  \item
    Since all checks have been passed, we finally insert the page at the
    relevant \texttt{dstva} address in the receiver's address space.
  \item
    \texttt{env\_ipc\_recving} is set to \texttt{0} since the message
    has been sent.
  \item
    Other variables such as \texttt{env\_ipc\_from} (environment that
    sent) and \texttt{env\_ipc\_value} (value that was sent) are also
    set.
  \item
    Finally the receiver is made runnable and the functions returns
    \texttt{0} (success).
  \end{itemize}
\item
  \textbf{\texttt{ipc\_recv()}:}

  \begin{itemize}
  \itemsep1pt\parskip0pt\parsep0pt
  \item
    This is a library function implemented in \texttt{lib/ipc.c}, so
    that C programs can call this function to make the system call to
    receive a message.
  \item
    \texttt{from\_env\_store} stores the sender's \texttt{envid}, and
    \texttt{perm\_store}, the permissions of the page sent.
  \item
    We set both these values in this function (copying
    \texttt{env\_ipc\_from} and \texttt{env\_ipc\_perm} into them) if
    the corresponding arguments are non-null.
  \item
    We also make use of the error returned by the system call
    \texttt{sys\_ipc\_recv()} here. If there is an error, the two
    \texttt{\_store} variables are set to \texttt{0} before returning
    the error.
  \item
    If there are no errors, \texttt{ipc\_recv()} returns the value it
    receives, which is stored in \texttt{env\_ipc\_value} of
    \texttt{thisenv}.
  \end{itemize}
\item
  \textbf{\texttt{ipc\_send()}:}

  \begin{itemize}
  \itemsep1pt\parskip0pt\parsep0pt
  \item
    This is a library function implemented in \texttt{lib/ipc.c}, so
    that C programs can call this function to make the system call to
    send a message.
  \item
    First we check if a page is to be sent, by checking if it is
    non-null. If it is \texttt{NULL}, it means that only the value is to
    be sent, so \texttt{pg} is set to \texttt{UTOP} to indicate to
    \texttt{ipc\_recv()} that no page is being sent.
  \item
    Next, we try to send the passed \texttt{pg} (with permissions
    \texttt{perm}) and \texttt{value} to the environment (with
    \texttt{envid} equal to \texttt{to\_env}).
  \item
    If the results is the error \texttt{-E\_IPC\_NOT\_RECV}, it means
    that \texttt{to\_env} does not wish to receive a message. Hence, the
    message isn't sent from \texttt{curenv} to \texttt{to\_env}, and we
    do a \texttt{sys\_yield()} to relinquish this environment's control
    over the CPU.
  \item
    Thus everytime control comes back to the current environment, the
    check is performed again and again till \texttt{to\_env} is in a
    state where it is willing to receive, in which case the result is
    \texttt{0} (success), and we exit out of the while loop.
  \item
    In case of any other error, the \texttt{while} condition is false,
    but the result in non-zero, so the kernel panics.
  \item
    Both \texttt{user/pingpong} and \texttt{user/primes} work
    successfully now.
  \item
    \texttt{make grade} successfully gives \texttt{75/75}.
  \end{itemize}
\end{enumerate}

\end{document}
