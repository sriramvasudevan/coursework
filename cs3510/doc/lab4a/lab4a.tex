\documentclass[]{article}
\usepackage[T1]{fontenc}
\usepackage{lmodern}
\usepackage{amssymb,amsmath}
\usepackage{ifxetex,ifluatex}
\usepackage{fixltx2e} % provides \textsubscript
% use microtype if available
\IfFileExists{microtype.sty}{\usepackage{microtype}}{}
% use upquote if available, for straight quotes in verbatim environments
\IfFileExists{upquote.sty}{\usepackage{upquote}}{}
\ifnum 0\ifxetex 1\fi\ifluatex 1\fi=0 % if pdftex
  \usepackage[utf8]{inputenc}
\else % if luatex or xelatex
  \usepackage{fontspec}
  \ifxetex
    \usepackage{xltxtra,xunicode}
  \fi
  \defaultfontfeatures{Mapping=tex-text,Scale=MatchLowercase}
  \newcommand{\euro}{€}
\fi
\usepackage[margin=0.8in]{geometry}
\ifxetex
  \usepackage[setpagesize=false, % page size defined by xetex
              unicode=false, % unicode breaks when used with xetex
              xetex]{hyperref}
\else
  \usepackage[unicode=true]{hyperref}
\fi
\hypersetup{breaklinks=true,
            bookmarks=true,
            pdfauthor={Sriram V - CS11B058},
            pdftitle={Lab 4 - Part A: Multiprocessor Support and Cooperative Multitasking},
            colorlinks=true,
            urlcolor=blue,
            linkcolor=magenta,
            pdfborder={0 0 0}}
\urlstyle{same}  % don't use monospace font for urls
\setlength{\parindent}{0pt}
\setlength{\parskip}{6pt plus 2pt minus 1pt}
\setlength{\emergencystretch}{3em}  % prevent overfull lines
\setcounter{secnumdepth}{0}

\title{Lab 4 - Part A: Multiprocessor Support and Cooperative Multitasking}
\author{Sriram V - CS11B058}
\date{}

\begin{document}
\maketitle

\subsection{Exercise 1}

\begin{itemize}
\itemsep1pt\parskip0pt\parsep0pt
\item
  Implemented \texttt{mmio\_map\_region()} in \texttt{kern/pmap.c}.
\item
  The passed \texttt{size} is rounded up and checks are made to ensure
  that the limit doesn't exceed \texttt{MMIOLIM}.
\item
  \texttt{boot\_map\_region()} is then called to map at
  \texttt{MMIOBASE} the passed physical address with the appropriate
  permissions.
\end{itemize}

\subsection{Exercise 2}

\begin{itemize}
\itemsep1pt\parskip0pt\parsep0pt
\item
  The only change that is made to \texttt{page\_init()} here is a
  special check for \texttt{MPENTRY \_PADDR}.
\item
  If the page is the page corresponding to this address, the
  \texttt{pp\_ref} is incremented for this page, and its link is set to
  \texttt{NULL}.
\item
  Now the AP bootstrap code can be safely copied and run at this
  physical address.
\item
  The code now successfully passes the updated
  \texttt{check\_page\_free\_list()}.
\end{itemize}

\subsubsection{Questions}

\begin{enumerate}
\def\labelenumi{\arabic{enumi}.}
\itemsep1pt\parskip0pt\parsep0pt
\item
  \textbf{Compare \texttt{kern/mpentry.S} side by side with
  \texttt{boot/boot.S}. Bearing in mind that \texttt{kern/mpentry.S} is
  compiled and linked to run above \texttt{KERNBASE} just like
  everything else in the kernel, what is the purpose of macro
  \texttt{MPBOOTPHYS}? Why is it necessary in \texttt{kern/mpentry.S}
  but not in \texttt{boot/boot.S}? In other words, what could go wrong
  if it were omitted in \texttt{kern/mpentry.S}? Hint: recall the
  differences between the link address and the load address that we have
  discussed in Lab 1.}
\end{enumerate}

\begin{itemize}
\itemsep1pt\parskip0pt\parsep0pt
\item
  MPBOOTPHYS(s) ((s) - mpentry\_start + MPENTRY\_PADDR) to calculate
  absolute addresses of its symbols, rather than relying on the linker
  to fill them
\item
  If it was exactly like boot.S, then the linker would put the symbols
  not where we want it, but elsewhere. We have control in this way.
\end{itemize}

\subsection{Exercise 3}

\begin{itemize}
\itemsep1pt\parskip0pt\parsep0pt
\item
  In \texttt{mem\_init\_mp()} in \texttt{kern/pmap.c}, for each of the
  \texttt{NCPU}s we use \texttt{boot\_map\_region()} to map into the
  kernel page directory the physical address of each CPU's stack.
\item
  The subtracted amount after each mapping is
  \texttt{KSTKSIZE + KSTKGAP}. This is because we need to account for
  both the stack size as well as the gap maintained between each stack
  (so that a stack overflow will not overwrite other stacks).
\item
  The code now passes \texttt{check\_pgdir()}.
\end{itemize}

\subsection{Exercise 4}

\begin{itemize}
\itemsep1pt\parskip0pt\parsep0pt
\item
  Here we modify the \texttt{trap\_init\_percpu()} function in
  \texttt{kern/trap.c} to handle multiple CPUs by primarily modifying
  the global \texttt{ts} variable into a local variable that is equal to
  the current CPU's Taskstate.
\item
  \texttt{ss0} remains as \texttt{GD\_KD} as the kernel's data segment
  won't change.
\item
  However, the stack used by each CPU is different, and consequently the
  \texttt{esp0} value is modified from \texttt{KSTACKTOP} to
  \texttt{KSTACKTOP - (KSTKSIZE + KSTKGAP) * cpu\_id}.
\item
  The index of the \texttt{GDT} and its offset vary from CPU to CPU as
  well, and thus the \texttt{TSS} of each CPU is calculated as
  \texttt{GD\_TSS0 + (cpu\_id \textless{}\textless{} 3)}.
\item
  The rest of the code remains the same.
\end{itemize}

\subsection{Exercise 5}

\begin{itemize}
\itemsep1pt\parskip0pt\parsep0pt
\item
  This exercise involves locking and unlocking the kernel at
  appropritate points to address race conditions when multiple CPUs run
  the kernel code simultaneously.
\item
  In \texttt{i386\_init()} in \texttt{kern/init.c}, the kernel lock is
  acquired before the APs are wokern up.
\item
  In \texttt{mp\_main()}, to ensure that only one CPU can enter the
  scheduler at a time, we lock the kernel before calling
  \texttt{sched\_yield()}.
\item
  In \texttt{trap()}, the lock is inserted if \texttt{tf\_cs} is equal
  to \texttt{3}.
\item
  In \texttt{env\_run()}, the kernel is unlocked right before popping
  the new environment's Trapframe.
\end{itemize}

\subsubsection{Questions}

\begin{enumerate}
\def\labelenumi{\arabic{enumi}.}
\setcounter{enumi}{1}
\itemsep1pt\parskip0pt\parsep0pt
\item
  \textbf{It seems that using the big kernel lock guarantees that only
  one CPU can run the kernel code at a time. Why do we still need
  separate kernel stacks for each CPU? Describe a scenario in which
  using a shared kernel stack will go wrong, even with the protection of
  the big kernel lock.}
\end{enumerate}

\begin{itemize}
\itemsep1pt\parskip0pt\parsep0pt
\item
  Can happen in the case of context switch. P1 pushes some data onto
  kernel stack to be accessed later. Critical data it is. Context switch
  to P2
\item
  Now P2 also pushes some critical data to access later. now context
  switch to p1. Now when P1 wants to access it's data that it pushed, it
  ends up having access to p2's data, a security vulnerability.
\end{itemize}

\subsection{Exercise 6}

\begin{itemize}
\itemsep1pt\parskip0pt\parsep0pt
\item
  Round robin scheduler is implemented in \texttt{sched\_yield()}.
\item
  The index is initialized to \texttt{-1}, so that if no current
  environment exists, the increment inside the loop will set index to
  zero.
\item
  Else, the index will be set inside the loop to the next environment.
\item
  We loop through all environments and search for an
  \texttt{ENV\_RUNNABLE} environment.
\item
  If no such environemnt is found, and the current environment is not
  null and is \texttt{ENV\_RUNNING}, then we \texttt{env\_run()} the
  current environment itself.
\item
  \texttt{yield.c} running on 3 environemnts now uses the scheduler
  succesfully.
\end{itemize}

\subsubsection{Questions}

\begin{enumerate}
\def\labelenumi{\arabic{enumi}.}
\setcounter{enumi}{2}
\itemsep1pt\parskip0pt\parsep0pt
\item
  \textbf{In your implementation of env\_run() you should have called
  lcr3(). Before and after the call to lcr3(), your code makes
  references (at least it should) to the variable e, the argument to
  env\_run. Upon loading the \%cr3 register, the addressing context used
  by the MMU is instantly changed. But a virtual address (namely e) has
  meaning relative to a given address context--the address context
  specifies the physical address to which the virtual address maps. Why
  can the pointer e be dereferenced both before and after the addressing
  switch?}
\end{enumerate}

\begin{itemize}
\itemsep1pt\parskip0pt\parsep0pt
\item
  This is because in every environment we map all the virtual addresses
  above UTOP (the kernel addresses) to be the same. Hence, we can
  continue to dereference the pointer before and after the context
  switch.
\item
  Because both page directories (kernel and user mode) has the same
  entries for the kernel addresses mapping.
\item
  That way reloading \%cr3 register makes no difference, since the `e'
  pointer will have a kernel address, which is correctly mapped in the
  just loaded page directory.
\end{itemize}

\begin{enumerate}
\def\labelenumi{\arabic{enumi}.}
\setcounter{enumi}{3}
\itemsep1pt\parskip0pt\parsep0pt
\item
  \textbf{Whenever the kernel switches from one environment to another,
  it must ensure the old environment's registers are saved so they can
  be restored properly later. Why? Where does this happen?}
\end{enumerate}

\begin{itemize}
\itemsep1pt\parskip0pt\parsep0pt
\item
  Called a context switch. Done so that an environment can resume from
  the pt where it switched. The Trapframe struct contains details of all
  regs for an env. env's pgdir and tf are changed when \texttt{env\_run}
  is called.
\end{itemize}

\subsection{Exercise 7}

\begin{itemize}
\itemsep1pt\parskip0pt\parsep0pt
\item
  The system calls implemented are primarily wrapper functions for
  already existing functions.
\item
  Hence, the only code that had to be written are the required sanity
  checks before calling these functions.
\item
  \texttt{dumbfork.c} now works successfully.
\end{itemize}

\end{document}
