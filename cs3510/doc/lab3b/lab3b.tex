\documentclass[]{article}
\usepackage[T1]{fontenc}
\usepackage{lmodern}
\usepackage{amssymb,amsmath}
\usepackage{ifxetex,ifluatex}
\usepackage{fixltx2e} % provides \textsubscript
% use microtype if available
\IfFileExists{microtype.sty}{\usepackage{microtype}}{}
% use upquote if available, for straight quotes in verbatim environments
\IfFileExists{upquote.sty}{\usepackage{upquote}}{}
\ifnum 0\ifxetex 1\fi\ifluatex 1\fi=0 % if pdftex
  \usepackage[utf8]{inputenc}
\else % if luatex or xelatex
  \usepackage{fontspec}
  \ifxetex
    \usepackage{xltxtra,xunicode}
  \fi
  \defaultfontfeatures{Mapping=tex-text,Scale=MatchLowercase}
  \newcommand{\euro}{€}
\fi
\usepackage[margin=0.8in]{geometry}
\ifxetex
  \usepackage[setpagesize=false, % page size defined by xetex
              unicode=false, % unicode breaks when used with xetex
              xetex]{hyperref}
\else
  \usepackage[unicode=true]{hyperref}
\fi
\hypersetup{breaklinks=true,
            bookmarks=true,
            pdfauthor={Sriram V - CS11B058},
            pdftitle={Lab 3 - Part B: Page Faults, Breakpoints Exceptions, and System Calls},
            colorlinks=true,
            urlcolor=blue,
            linkcolor=magenta,
            pdfborder={0 0 0}}
\urlstyle{same}  % don't use monospace font for urls
\setlength{\parindent}{0pt}
\setlength{\parskip}{6pt plus 2pt minus 1pt}
\setlength{\emergencystretch}{3em}  % prevent overfull lines
\setcounter{secnumdepth}{0}

\title{Lab 3 - Part B: Page Faults, Breakpoints Exceptions, and System Calls}
\author{Sriram V - CS11B058}
\date{}

\begin{document}
\maketitle

\subsection{Exercise 5}

\begin{itemize}
\itemsep1pt\parskip0pt\parsep0pt
\item
  Modified \texttt{trap\_dispatch()} in \texttt{kern/trap.c} to handle
  page fault exceptions.
\item
  We check whether \texttt{tf-\textgreater{}tf\_trapno} is equal to
  \texttt{T\_PGFLT} and call \texttt{page\_fault\_handler(tf)} if they
  are. (\texttt{lines 159 - 161})
\item
  \texttt{make grade} now succeeds on the \texttt{faultread},
  \texttt{faultreadkernel}, \texttt{faultwrite}, and
  \texttt{faultwritekernel} tests.
\end{itemize}

\subsection{Exercise 6}

\begin{itemize}
\itemsep1pt\parskip0pt\parsep0pt
\item
  Modified \texttt{trap\_dispatch()} in \texttt{kern/trap.c} to handle
  breakpoint exceptions.
\item
  We check if \texttt{tf-\textgreater{}tf\_trapno} is \texttt{T\_BRKPT}
  and call \texttt{monitor(tf)} if it is. (\texttt{lines 163 - 165})
\item
  We also change the privilege level from \texttt{0} to \texttt{3} in
  the IDT for the breakpoint's trap handler entry, in
  \texttt{trap\_init()}. (\texttt{line 76})
\item
  \texttt{make grade} now succeeds on the \texttt{breakpoint} test.
\end{itemize}

\subsubsection{Questions}

\begin{enumerate}
\def\labelenumi{\arabic{enumi}.}
\setcounter{enumi}{2}
\itemsep1pt\parskip0pt\parsep0pt
\item
  \textbf{The break point test case will either generate a break point
  exception or a general protection fault depending on how you
  initialized the break point entry in the IDT (i.e., your call to
  \texttt{SETGATE} from \texttt{trap\_init}). Why? How do you need to
  set it up in order to get the breakpoint exception to work as
  specified above and what incorrect setup would cause it to trigger a
  general protection fault?}
\end{enumerate}

\begin{itemize}
\itemsep1pt\parskip0pt\parsep0pt
\item
  If the break point entry is set up in the IDT to be run only with
  kernel privileges (\texttt{DPL=0}), a general protection fault is
  triggered when it is called from the user level due to a privilege
  level violation.
\item
  However if the breakpoint entry is set with \texttt{DPL=3} (user mode)
  in the IDT, then it can be called from the user level.
\end{itemize}

\begin{enumerate}
\def\labelenumi{\arabic{enumi}.}
\setcounter{enumi}{3}
\itemsep1pt\parskip0pt\parsep0pt
\item
  \textbf{What do you think is the point of these mechanisms,
  particularly in light of what the \texttt{user/softint} test program
  does?}
\end{enumerate}

\begin{itemize}
\itemsep1pt\parskip0pt\parsep0pt
\item
  We want only a few exceptions (such as the system call and breakpoint
  exceptions) to be accessible by the user. The user musn't have access
  to all 256 exceptions for security purposes.
\item
  Thus, we set the exceptions to be accessed by the user to
  \texttt{DPL=3} in the IDT, and \texttt{DPL=0} for the rest.
\item
  In \texttt{user/softint}, the user tries to generate a page fault. He
  could get the kernel to run malicious code by pushing this code on to
  the stack and calling the page fault, so that the interrupt service
  routine pops the malicious code and executes it. The page fault could
  also be used to allocate additional physical memory (a page fault
  would imply that the page doesn't exist in the memory, so the kernel
  would allocate additional physical memory assuming it would have to
  load in new pages) for the process -- something no user should be able
  to do from user mode.
\end{itemize}

\subsection{Exercise 7}

\begin{itemize}
\itemsep1pt\parskip0pt\parsep0pt
\item
  To handle the system call (exception number \texttt{48}), I had to add
  in trap handlers (that don't take any error codes -
  \texttt{TRAPHANDLER\_NOEC()}) for all exceptions from \texttt{20} to
  \texttt{48}, since I use an array for the trap handlers (Lab 3 - Part
  A's challenge). By adding in these extra trap handlers, I can index
  into the array easily, rather than remembering which index maps to
  which exception number.
\item
  I then created corresponding entries for all these trap handlers in
  the IDT in \texttt{kern/trap.c}, by increasing my iterations from
  \texttt{20} (done in Part A) to \texttt{49}.
\item
  The privilege level for the \texttt{T\_SYSCALL} trap handler entry in
  the IDT is changed to \texttt{3} -- user level. (\texttt{line 79} in
  \texttt{kern/trap.c})
\item
  \texttt{trap\_dispatch()} is modified to handle \texttt{T\_SYSCALL}.
  The \texttt{syscall()} function is called, and the \texttt{EAX},
  \texttt{EDX}, \texttt{ECX}, \texttt{EBX}, \texttt{EDI} and
  \texttt{ESI} register values (present in the \texttt{Trapframe}
  structure) are passed to it.
\item
  Its return value is stored back in the \texttt{EAX} register of the
  \texttt{Trapframe} structure.
\item
  The \texttt{syscall()} function is modified in \texttt{kern/syscall.c}
  to handle the various system calls (the system call numbers are
  specified in \texttt{inc/syscall.h}).
\item
  A \texttt{switch-case} is used to compare \texttt{syscallno} with the
  various system call numbers, and call the appropriate function.
\item
  Since \texttt{sys\_cputs()} doesn't return anything, we return
  \texttt{0} instead.
\item
  If the system call number is invalid, an error message is printed and
  \texttt{-E\_INVAL} is returned.
\item
  Running \texttt{make run-hello-nox} prints \texttt{hello, world} on
  the console and causes a page fault in user mode.
\item
  \texttt{make grade} now succeeds on the \texttt{testbss} test.
\end{itemize}

\subsection{Exercise 8}

\begin{itemize}
\itemsep1pt\parskip0pt\parsep0pt
\item
  The \texttt{user/hello.c} program page faults because it tries to
  access the current environment's user id
  (\texttt{thisenv-\textgreater{}env\_id}). However, \texttt{thisenv}
  has been set to \texttt{NULL} in \texttt{lib/libmain.c}.
\item
  We rectify this by getting the current environment's system ID using
  the \texttt{sys\_getenvid()} system call.
\item
  We make use of the \texttt{ENVX()} macro to get the corresponding
  environment entry in the \texttt{envs{[}{]}} array.
\item
  \texttt{thisenv} is now made to point to this object.
\item
  Booting into the kernel now prints \texttt{hello, world} as well as
  \texttt{i am environment 00001000}.
\item
  Since the kernel currently supports only one user environment and
  \texttt{user/hello.c} calls \texttt{sys\_env\_destroy()}, the kernel
  reports that the only environment has been destroyed and drops into
  the kernel monitor.
\item
  \texttt{make grade} now succeeds on the \texttt{hello} test.
\end{itemize}

\subsection{Exercise 9}

\begin{itemize}
\itemsep1pt\parskip0pt\parsep0pt
\item
  The \texttt{page\_fault\_handler()} function in \texttt{kern/trap.c}
  is modified to panic if a page fault happens in kernel mode.
\item
  We check if the low bits of \texttt{tf-\textgreater{}tf\_cs} is not
  equal to the user's DPL (\texttt{0x3}).
\item
  \texttt{user\_mem\_check()} in \texttt{kern/pmap.c} is then
  implemented. The passed permissions \texttt{perm} is modified to
  include \texttt{PTE\_P}.
\item
  The start virtual address \texttt{va} is rounded down, and
  \texttt{va+len} is rounded up and set as the end address.
\item
  If the address accessed is greater than \texttt{ULIM}, we jump to
  \texttt{bad}.
\item
  Now we try to get the corresponding page entry in the environment's
  page table. If it doesn't exist, we once again jump to \texttt{bad}.
\item
  If it does exist, we check if the set permissions are equal to the
  permissions passed to the function. If it isn't, we jump to
  \texttt{bad}.
\item
  These checks are performed for all pages in the range
  \texttt{{[}va, va+len)}.
\item
  If the user program can access this range of addresses, the function
  returns \texttt{0}.
\item
  \texttt{bad} is a label below which we write the code to handle all
  the errors.
\item
  We set \texttt{user\_mem\_check\_addr} to the first erroneous virtual
  address. This is done by initialising this variable to \texttt{va},
  and if the current address being checked is greater than \texttt{va},
  we set \texttt{user\_mem\_check\_addr} to this value (since this is
  the first address at which the error occured).
\item
  We now return \texttt{-E\_FAULT} as an error has occured.
\item
  \texttt{debuginfo\_eip()} is modified in \texttt{kern/kdebug.c} to
  call \texttt{user\_mem\_assert()} on \texttt{usd}, \texttt{stabs} and
  \texttt{stabstr}, to check whether they have \texttt{PTE\_U}
  permissions.
\item
  \texttt{make run-breakpoint-nox} now allows us to enter the kernel
  monitor and use the \texttt{backtrace} command.
\item
  We see the backtrace traverse into lib/libmain.c before the kernel
  panics with a page fault.
\end{itemize}

\subsection{Exercise 10}

\begin{itemize}
\item
  We now run \texttt{make run-evilhello-nox}. The environment gets
  destroyed, as is required by this exercise.
\item
  The output observed is:

\begin{verbatim}
[00001000] user_mem_check assertion failure for va f010000c
[00001000] free env 00001000
Destroyed the only environment - nothing more to do!
\end{verbatim}
\item
  All the \texttt{make grade} tests pass now.
\end{itemize}

\end{document}
