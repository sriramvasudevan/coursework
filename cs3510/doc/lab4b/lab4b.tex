\documentclass[]{article}
\usepackage[T1]{fontenc}
\usepackage{lmodern}
\usepackage{amssymb,amsmath}
\usepackage{ifxetex,ifluatex}
\usepackage{fixltx2e} % provides \textsubscript
% use microtype if available
\IfFileExists{microtype.sty}{\usepackage{microtype}}{}
% use upquote if available, for straight quotes in verbatim environments
\IfFileExists{upquote.sty}{\usepackage{upquote}}{}
\ifnum 0\ifxetex 1\fi\ifluatex 1\fi=0 % if pdftex
  \usepackage[utf8]{inputenc}
\else % if luatex or xelatex
  \usepackage{fontspec}
  \ifxetex
    \usepackage{xltxtra,xunicode}
  \fi
  \defaultfontfeatures{Mapping=tex-text,Scale=MatchLowercase}
  \newcommand{\euro}{€}
\fi
\usepackage[margin=0.8in]{geometry}
\ifxetex
  \usepackage[setpagesize=false, % page size defined by xetex
              unicode=false, % unicode breaks when used with xetex
              xetex]{hyperref}
\else
  \usepackage[unicode=true]{hyperref}
\fi
\hypersetup{breaklinks=true,
            bookmarks=true,
            pdfauthor={Sriram V - CS11B058},
            pdftitle={Lab 4 - Part B: Copy-on-Write Fork},
            colorlinks=true,
            urlcolor=blue,
            linkcolor=magenta,
            pdfborder={0 0 0}}
\urlstyle{same}  % don't use monospace font for urls
\setlength{\parindent}{0pt}
\setlength{\parskip}{6pt plus 2pt minus 1pt}
\setlength{\emergencystretch}{3em}  % prevent overfull lines
\setcounter{secnumdepth}{0}

\title{Lab 4 - Part B: Copy-on-Write Fork}
\author{Sriram V - CS11B058}
\date{}

\begin{document}
\maketitle

\subsection{Exercise 8}

\begin{itemize}
\itemsep1pt\parskip0pt\parsep0pt
\item
  Implemented \texttt{sys\_env\_set\_pgfault\_upcall()} in
  \texttt{kern/syscall.c}.
\item
  User environments register a page fault handler entrypoint with the
  JOS kernel using this system call.
\item
  The information is stored in \texttt{env\_pgfault\_upcall} fo the
  \texttt{Env} structure.
\item
  The system call takes the Environment ID and the handler as its
  arguments.
\item
  We assert that the handler \texttt{func} is not \texttt{NULL}, and
  access the relevant \texttt{Env} structure associated with
  \texttt{envid} using the \texttt{envid2env()} function defined in
  \texttt{kern/env.c}.
\item
  \texttt{checkperm} is set to \texttt{1} above, as it is a `dangerous'
  system call.
\item
  If no such environment exists, we return \texttt{-E\_BAD\_ENV}
\item
  Else, we set the \texttt{env\_pgfault\_upcall} variable of this
  environment as \texttt{func}.
\item
  The function returns \texttt{0} on success.
\end{itemize}

\subsection{Exercise 9}

\begin{itemize}
\itemsep1pt\parskip0pt\parsep0pt
\item
  Here, \texttt{page\_fault\_handler()} in \texttt{kern/trap.c} is
  modified to handle page faults from the user mode.
\item
  Since the function is required to dispatch page faults to the
  user-mode handler, we check if \texttt{pgfault\_upcall} exists for
  \texttt{curenv}. We also check if the environment's stack pointer
  (stored in \texttt{tf-\textgreater{}tf\_esp}) is within the stack
  limits. Else, we destroy the enironment.
\item
  If the \texttt{esp} value is between the range
  \texttt{UXSTACKTOP-PGSIZE} and \texttt{UXSTACKTOP-1} inclusive, then
  the page fault handler itself has faulted. In such a case, a 32-bit
  empty word is pushed, and then a \texttt{struct UTrapframe}.
\item
  Otherwise the \texttt{struct UTrapframe} is directly pushed onto the
  stack.
\item
  The \texttt{UTrapframe} ovject's variables are all set to the current
  \texttt{tf}'s values.
\item
  Then \texttt{tf}'s \texttt{eip} is set to \texttt{curenv}'s
  \texttt{env\_pgfault\_upcall}, so that the call to \texttt{env\_run()}
  will start from that point.
\item
  The \texttt{esp} value of \texttt{tf} is now updated to point to the
  beginning of the \texttt{UTrapframe} object, as it is the topmost on
  the stack.
\item
  \texttt{env\_run()} is then called on \texttt{curenv}.
\end{itemize}

\subsection{Exercise 10}

\begin{itemize}
\itemsep1pt\parskip0pt\parsep0pt
\item
  The \texttt{\_pgfault\_upcall} routine in \texttt{lib/pfentry.S} is
  wgere the kernel redirects control whenever a page fault is caused in
  user space.
\item
  The stack pointer \texttt{esp} currently points to the exception
  stack, which we temporarily save in \texttt{eax}.
\item
  The \texttt{eip} value (where we have to return to) is loaded into
  \texttt{ebx}.
\item
  We then switch to the trap-time stack (\texttt{utf\_esp}), and push
  the return address (now in \texttt{ebx}) onto the top of this stack.
\item
  The trap-time stack pointer is also updated in the exception stack,
  since a new value has been pushed onto it.
\item
  We now switch back to the exception stack, skip the fault address and
  the error, and restore all the trap-time registers
  (\texttt{utf\_regs}).
\item
  The return address is skipped over, and the \texttt{eflags} are also
  restored.
\item
  The top of the exception stack now points to the trap-time stack
  pointer. The value is popped into \texttt{esp} to switch to it.
\item
  \texttt{ret} is called to return to the address on the top of the
  trap-time stack, which is the \texttt{eip} value we pushed at the
  start of this code segment.
\item
  Thus, this code successfully restores the state at which the fault
  occured, and re-runs the code that caused the fault.
\end{itemize}

\subsection{Exercise 11}

\begin{itemize}
\itemsep1pt\parskip0pt\parsep0pt
\item
  The \texttt{set\_pgfault\_handler()} method in \texttt{lib/pgfault.c}
  is how the system call to register the page fault handler with the JOS
  kernel is implemented.
\item
  If \texttt{\_pgfault\_handler} is \texttt{0}, it is the first time a
  handler is being registered. In such a case, we need to allocate an
  exception stack of size \texttt{PGSIZE} at \texttt{UXSTACKTOP}, and
  also tell the kernel to call the \texttt{\_pgfault\_upcall} routine
  (implemented in the previous exercise) whenever a page fault occurs.
\item
  This is done using the system calls \texttt{sys\_page\_alloc()} and
  \texttt{sys\_env\_set\_pgfault\_upcall()} respectively.
\item
  The variable \texttt{\_pgfault\_handler} is then set to the
  \texttt{handler} that was passed, so that the assembly routine can
  call it when needed.
\end{itemize}

\subsection{Exercise 12}

\begin{enumerate}
\def\labelenumi{\arabic{enumi}.}
\itemsep1pt\parskip0pt\parsep0pt
\item
  \textbf{\texttt{fork()}:}

  \begin{itemize}
  \itemsep1pt\parskip0pt\parsep0pt
  \item
    Until now, a \texttt{fork()} resulted in all the pages being copied
    from the parent to the child environment, but this causes a slowdown
    and also may not necessarily be used always, as an \texttt{exec()}
    may be called which replaces the child environment's address space.
  \item
    Hence, the new implementation of \texttt{fork()} copies page
    mappings instead of pages.
  \item
    A separate copy is created only when either environment modifies the
    pages -- this is called copy-on-write.
  \item
    First, \texttt{pgfault()} is set as the page fault handler for
    \texttt{fork()}.
  \item
    Then a new environment is spawned using \texttt{sys\_exofork()}.
  \item
    The pages of the old environment are iterated over, and the pages
    are duplicated into the new environment by calling
    \texttt{duppage()}, apart from the last page.
  \item
    The final page (just below \texttt{UXSTACKTOP}) is allocated for the
    exception stack, and the \texttt{pgfault\_upcall} values are also
    copied.
  \item
    Finally, the new environment is set to \texttt{ENV\_RUNNABLE} and
    its ID is returned.
  \end{itemize}
\item
  \textbf{\texttt{pgfault()}:}

  \begin{itemize}
  \itemsep1pt\parskip0pt\parsep0pt
  \item
    This function is the page fault handler for the \texttt{fork()}
    process.
  \item
    We check that the fault is a write (\texttt{FEC\_WR}) and that the
    page is marked as \texttt{PTE\_COW}.
  \item
    A new page is then allocated at \texttt{PFTEMP}, and the contents of
    the old faulting page at \texttt{addr} are copied into this page.
  \item
    Finally the new page is mapped with read/write permissions at the
    old address, in place of the old read-only mapping.
  \end{itemize}
\item
  \textbf{\texttt{duppage()}}:

  \begin{itemize}
  \itemsep1pt\parskip0pt\parsep0pt
  \item
    First, a check is performed to see if the page is writable or
    copy-on-write.
  \item
    If it is, the page to be duplicated (present in the current
    environment \texttt{0}'s address space) is mapped to the same
    virtual address, but in \texttt{envid}'s address space, and with the
    User, Present and COW bits set.
  \item
    Next, the permissions of duplicated page are also changed to
    copy-on-write.
  \item
    To do this, the page at \texttt{va} in \texttt{0}'s address space is
    mapped to the same address in the same environment, but with the
    appropriate permission bits set.
  \item
    Finally, on success of this entire procedure, \texttt{0} is
    returned.
  \end{itemize}
\end{enumerate}

\end{document}
