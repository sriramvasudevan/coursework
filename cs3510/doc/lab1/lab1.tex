\documentclass[]{article}
\usepackage[T1]{fontenc}
\usepackage[margin=0.8in]{geometry}
\usepackage{lmodern}
\usepackage{amssymb,amsmath}
\usepackage{ifxetex,ifluatex}
\usepackage{fixltx2e} % provides \textsubscript
% use microtype if available
\IfFileExists{microtype.sty}{\usepackage{microtype}}{}
% use upquote if available, for straight quotes in verbatim environments
\IfFileExists{upquote.sty}{\usepackage{upquote}}{}
\ifnum 0\ifxetex 1\fi\ifluatex 1\fi=0 % if pdftex
  \usepackage[utf8]{inputenc}
\else % if luatex or xelatex
  \usepackage{fontspec}
  \ifxetex
    \usepackage{xltxtra,xunicode}
  \fi
  \defaultfontfeatures{Mapping=tex-text,Scale=MatchLowercase}
  \newcommand{\euro}{€}
\fi
\ifxetex
  \usepackage[setpagesize=false, % page size defined by xetex
              unicode=false, % unicode breaks when used with xetex
              xetex]{hyperref}
\else
  \usepackage[unicode=true]{hyperref}
\fi
\hypersetup{breaklinks=true,
            bookmarks=true,
            pdfauthor={},
            pdftitle={},
            colorlinks=true,
            urlcolor=blue,
            linkcolor=magenta,
            pdfborder={0 0 0}}
\urlstyle{same}  % don't use monospace font for urls
\setlength{\parindent}{0pt}
\setlength{\parskip}{6pt plus 2pt minus 1pt}
\setlength{\emergencystretch}{3em}  % prevent overfull lines
\setcounter{secnumdepth}{0}

\title{Lab 1: PC Bootstrap and GCC Calling Conventions}
\author{Sriram V - CS11B058}
\date{}

\begin{document}
\maketitle

\subsection{Exercise 1}

\begin{itemize}
\item
  Read and understood the AT\&T syntax from Brennan's Guide to Inline
  Assembly.
\item
  Extended inline assembly code takes the
  form\\\texttt{asm ( "statements" : output\_registers : input\_registers : clobbered\_registers);}
\item
  We use the following register loading codes in this function:

\begin{verbatim}
a        eax
b        ebx
c        ecx
d        edx
S        esi
D        edi
I        constant value (0 to 31)
q,r      dynamically allocated register (see below)
g        eax, ebx, ecx, edx or variable in memory
A        eax and edx combined into a 64-bit integer (use long longs)
\end{verbatim}
\end{itemize}

\subsection{Exercise 2}

\begin{itemize}
\itemsep1pt\parskip0pt\parsep0pt
\item
  When the BIOS runs, it sets up an interrupt descriptor table and
  initializes various devices such as the VGA display.
\item
  This is where the \texttt{Starting SeaBIOS} message in the QEMU window
  comes from.
\item
  After initializing the PCI bus and all the important devices the BIOS
  knows about, it searches for a bootable device such as a floppy, hard
  drive, or CD-ROM.
\item
  Eventually, when it finds a bootable disk, the BIOS reads the boot
  loader from the disk and transfers control to it.
\end{itemize}

\subsection{Exercise 3}

\begin{itemize}
\itemsep1pt\parskip0pt\parsep0pt
\item
  We set a breakpoint at \texttt{0x7c00} (\texttt{b *0x7c00}). The boot
  sector will be loaded here.
\item
  First the code is assembled for 16-bit mode (line \texttt{14} in
  \texttt{boot/boot.S} - \texttt{.code16})
\item
  Interrupts are cleared with \texttt{cli}.
\item
  \texttt{\%ds}, \texttt{\%es} and \texttt{\%ss} are all set to 0.
\item
  A20 is then enabled.
\item
  GDT is loaded at \texttt{line 48} in \texttt{boot/boot.S}
\item
  \texttt{CR0} is set to enable Protected mode.
\item
  We now jump to the next instruction, but in the 32-bit code segment.
  doing \texttt{si} in \texttt{gdb} tells us that
  \texttt{ljmp \$0xb866,\$0x87c32} does this. CS is set here
  (\texttt{\$PROT\_MODE\_CSEG}).
\item
  Lines \texttt{60} to \texttt{65} set the other registers in the GDT.
\item
  Now the stack pointer is set up, and the call to \texttt{bootmain()}
  is made.
\item
  \texttt{readseg()} is called to read the first page off the disk. A
  check is done if it is a valid ELF.
\item
  ph is the program header segment. The for loop reads the segments one
  the total number of segments present (\texttt{e\_phnum})
\item
  In \texttt{readseg()} the segments and offsets are translated into
  sector numbers and necessary round downs are made.
\item
  Direct mapping to memory is done since paging isn't enabled.
\item
  \texttt{readsect()} reads in data from the Hard Disk by using the
  sectors and the offsets.
\item
  Once the for loop in \texttt{bootmain()} is done, it uses the function
  pointer (to the entry point of the kernel that has just been loaded)
  to step into the beginning of the kernel code.
\end{itemize}

\subsection{Questions}

\begin{enumerate}
\def\labelenumi{\arabic{enumi}.}
\itemsep1pt\parskip0pt\parsep0pt
\item
  \textbf{At what point does the processor start executing 32-bit code?
  What exactly causes the switch from 16- to 32-bit mode?}
\end{enumerate}

\begin{itemize}
\itemsep1pt\parskip0pt\parsep0pt
\item
  Before \texttt{line 55} in \texttt{boot/boot.S}, where a long jump to
  32-bit code happens. The switch is due to setting \texttt{CR0}'s first
  bit to 1.
\end{itemize}

\begin{enumerate}
\def\labelenumi{\arabic{enumi}.}
\setcounter{enumi}{1}
\itemsep1pt\parskip0pt\parsep0pt
\item
  \textbf{What is the last instruction of the boot loader executed, and
  what is the first instruction of the kernel it just loaded?}
\end{enumerate}

\begin{itemize}
\itemsep1pt\parskip0pt\parsep0pt
\item
  Last instruction boot loader executes is the call to jump to the
  kernel's entry point (\texttt{0x100000c}). The instruction is located
  at \texttt{0x7d65} and we see that the jump is to the address stored
  at \texttt{0x10018}.
\item
  First kernel instrucion is \texttt{movw \$0x1234,0x472} (warm boot)
\end{itemize}

\begin{enumerate}
\def\labelenumi{\arabic{enumi}.}
\setcounter{enumi}{2}
\itemsep1pt\parskip0pt\parsep0pt
\item
  \textbf{Where is the first instruction in the kernel?}
\end{enumerate}

\begin{itemize}
\itemsep1pt\parskip0pt\parsep0pt
\item
  At \texttt{0x100000c} - the \texttt{.globl entry} in
  \texttt{kern/entry.S}.
\end{itemize}

\begin{enumerate}
\def\labelenumi{\arabic{enumi}.}
\setcounter{enumi}{3}
\itemsep1pt\parskip0pt\parsep0pt
\item
  \textbf{How does the boot loader decide how many sectors it must read
  in order to fetch the entire kernel from disk? Where does it find this
  information?}
\end{enumerate}

\begin{itemize}
\itemsep1pt\parskip0pt\parsep0pt
\item
  The ELF header specifies the offset for the program headers, and how
  many are there. The program headers contain the address of the program
  segment loaded into the ram, its size and the offset from the
  beginning of the file at which the first byte of the segment is
  present.
\end{itemize}

\subsection{Exercise 4}

\begin{itemize}
\itemsep1pt\parskip0pt\parsep0pt
\item
  Read relevant portions of THe C Programming Language by Kernighan and
  Richie.
\item
  An important concept here is that of pointer arithmetic. If
  \texttt{int *p = (int*)100}, then \texttt{(int)p + 1} would evaluate
  as 101, but \texttt{(int)(p + 1)} would be equal to 104.
\end{itemize}

\subsection{Exercise 5}

\begin{itemize}
\itemsep1pt\parskip0pt\parsep0pt
\item
  The parameter \texttt{-Ttext} in \texttt{boot/Makefrag} is modified
  from \texttt{0x7C00} to \texttt{0x7F00}. The code is recompiled and
  \texttt{gdb} is restarted.
\item
  The error occurs at \texttt{line 55} in \texttt{boot/boot.S} -- The
  long jump into the 32-bit code segment.
\item
  This happens because the \texttt{Ttext} parameter tells the assembler
  to link to the \texttt{0x7F00} address (ie., the code will execute
  from there), but the BIOS loads the code at \texttt{0x7C000} and the
  GDT data at \texttt{0x7c64}. But since the link address is
  \texttt{0x7f00}, at the \texttt{ljmp}, when the GDT is accessed, we
  end up looking for it at \texttt{0x7f64} instead.
\end{itemize}

\subsection{Exercise 6}

\begin{itemize}
\item
  When the BIOS enters the bootloader, using \texttt{x/8x 0x00100000}
  gives:

\begin{verbatim}
0x100000:   0x00000000  0x00000000  0x00000000  0x00000000
0x100010:   0x00000000  0x00000000  0x00000000  0x00000000
\end{verbatim}
\item
  When the bootloader enters the kernel, the same command gives:

\begin{verbatim}
0x100000:   0x1badb002  0x00000000  0xe4524ffe  0x7205c766
0x100010:   0x34000004  0x0000b812  0x220f0011  0xc0200fd8
\end{verbatim}
\item
  They are different because when the BIOS enters the bootloader,
  nothing has been loaded here.
\item
  However, the bootloader loads the kernel at \texttt{0x00100000} (The
  load address ``LMA'', when we do \texttt{objdump -h obj/kern/kernel})
\item
  Hence, we see values loaded into these addresses at the second
  breakpoint.
\end{itemize}

\subsection{Exercise 7}

\begin{itemize}
\item
  \texttt{movl \%eax, \%cr0} is the line that enables Paging.
\item
  \texttt{x/8x 0x00100000} gives us:

\begin{verbatim}
0x100000:   0x1badb002  0x00000000  0xe4524ffe  0x7205c766
0x100010:   0x34000004  0x0000b812  0x220f0011  0xc0200fd8
\end{verbatim}
\item
  \texttt{x/8x 0xf0100000} gives:

\begin{verbatim}
0xf0100000 <_start+4026531828>: 0xffffffff  0xffffffff  0xffffffff  0xffffffff
0xf0100010 <entry+4>:   0xffffffff  0xffffffff  0xffffffff  0xffffffff
\end{verbatim}
\item
  After single-stepping over the \texttt{movl} command, we see that the
  8 words at \texttt{0x00100000} are the same, but now the addresses at
  \texttt{0xf0100000} are mapped to \texttt{0x00100000}. Hence doing
  \texttt{x/8x 0xf0100000} gives us:

\begin{verbatim}
0xf0100000 <_start+4026531828>: 0x1badb002  0x00000000  0xe4524ffe  0x7205c766
0xf0100010 <entry+4>:   0x34000004  0x0000b812  0x220f0011  0xc0200fd8
\end{verbatim}
\item
  If the \texttt{movl} line is commented out (ie., Paging is disabled)
  and the code is recompiled, then execution fails at \texttt{line 68}
  in \texttt{kern/entry.S} (\texttt{jmp *\%eax}). This is because
  \texttt{relocated} is linked at the address \texttt{0xf010002f}
  (``VMA'' is \texttt{0xf0100000} for the kernel), and this addreass is
  loaded into \texttt{\%eax}.
\item
  But since paging wasn't enabled, the mapping failed.
\end{itemize}

\subsection{Exercise 8}

\begin{itemize}
\item
  The following code is added at lines 209 - 211 to enable printing of
  octal numbers.

\begin{verbatim}
num = getuint(&ap, lflag);
base = 8;
goto number;
\end{verbatim}
\end{itemize}

\subsection{Questions}

\begin{enumerate}
\def\labelenumi{\arabic{enumi}.}
\itemsep1pt\parskip0pt\parsep0pt
\item
  \textbf{Explain the interface between \texttt{printf.c} and
  \texttt{console.c}. Specifically, what function does
  \texttt{console.c} export? How is this function used by
  \texttt{printf.c}?}
\end{enumerate}

\begin{itemize}
\itemsep1pt\parskip0pt\parsep0pt
\item
  \texttt{printf.c} is able to access the \texttt{console.c} functions
  through their function declarations in \texttt{inc/stdio.h}.
\item
  printf.c defines the high-level printing functions used by the kernel,
  while console.c defines low-level functions used to perform some kinds
  of console I/O, like keyboard input support, serial I/O, parellel port
  and display output.
\item
  Specifically, \texttt{console.c} exports \texttt{cputchar()} that
  \texttt{printf.c} uses in \texttt{putch().}
\end{itemize}

\begin{enumerate}
\def\labelenumi{\arabic{enumi}.}
\setcounter{enumi}{1}
\item
  \textbf{Explain the following code from \texttt{console.c}:}

\begin{verbatim}
1      if (crt_pos >= CRT_SIZE) {
2              int i;
3              memcpy(crt_buf, crt_buf + CRT_COLS, (CRT_SIZE - CRT_COLS) * sizeof(uint16_t));
4              for (i = CRT_SIZE - CRT_COLS; i < CRT_SIZE; i++)
5                      crt_buf[i] = 0x0700 | ' ';
6              crt_pos -= CRT_COLS;
7      }
\end{verbatim}
\end{enumerate}

\begin{itemize}
\itemsep1pt\parskip0pt\parsep0pt
\item
  When the console buffer is full, it means that all the lines on the
  screen are filled. Hence, for new lines to be printed, the screen must
  scroll down. This code copies the content after the first
  \texttt{CRT\_COLS} onwards onto the display, thus removing the topmost
  row. It then blanks out the bottom-most row (the for loop goes to each
  column element in the last row). Then it sets the position of the next
  output to the first column element of the last, blanked-out row. This
  has resulted in a scrolling down of one line.
\end{itemize}

\begin{enumerate}
\def\labelenumi{\arabic{enumi}.}
\setcounter{enumi}{2}
\item
  \textbf{Trace the code execution step-by-step}

  \begin{enumerate}
  \def\labelenumii{\roman{enumii}.}
  \itemsep1pt\parskip0pt\parsep0pt
  \item
    \textbf{In the call to cprintf(), to what does \texttt{fmt} point?
    To what does \texttt{ap} point?}
  \end{enumerate}

  \begin{itemize}
  \itemsep1pt\parskip0pt\parsep0pt
  \item
    \texttt{fmt} points the address (\texttt{0xf0101a69}) of the string
    to be parsed:\\ \texttt{"x \%d, y \%x, z \%d\textbackslash{}n"}
  \item
    \texttt{ap} points to the addresss (\texttt{0xf010ffe4}) of the
    value of x - \texttt{0x00000001}
  \end{itemize}

  \begin{enumerate}
  \def\labelenumii{\roman{enumii}.}
  \setcounter{enumii}{1}
  \itemsep1pt\parskip0pt\parsep0pt
  \item
    \textbf{List (in order of execution) each call to
    \texttt{cons\_putc}, \texttt{va\_arg}, and \texttt{vcprintf}. For
    \texttt{cons\_putc}, list its argument as well. For
    \texttt{va\_arg}, list what \texttt{ap} points to before and after
    the call. For \texttt{vcprintf} list the values of its two
    arguments.}
  \end{enumerate}

  \begin{itemize}
  \itemsep1pt\parskip0pt\parsep0pt
  \item
    Order of calling: \texttt{vcprintf} \textgreater{} \texttt{va\_arg}
    \textgreater{} \texttt{cons\_putc}
  \item
    The argument for cons\_putc is the variable to be printed on the
    screen.
  \item
    \texttt{ap} points to the address of the first argument in the
    argument list before the call, and to the next argument in the list
    after the call.
  \item
    \texttt{fmt} points the address (\texttt{0xf0101a69}) of the string
    to be parsed: \texttt{"x \%d, y \%x, z \%d\textbackslash{}n"}
  \item
    \texttt{ap} points to the addresss (\texttt{0xf010ffe4}) of the
    value of x - \texttt{0x00000001}
  \end{itemize}
\item
  \textbf{Run the following code.}

\begin{verbatim}
unsigned int i = 0x00646c72;
cprintf("H%x Wo%s", 57616, &i);
\end{verbatim}

  \textbf{What is the output? Explain how this output is arrived.}
\end{enumerate}

\begin{itemize}
\itemsep1pt\parskip0pt\parsep0pt
\item
  The output is ``He110 World''
\item
  This is because 57616 in Decimal is e110 in Hex.
\item
  Since x86 uses the little-endian system, the bytes are accessed from
  right to left. so \texttt{0x72}=`r' is printed first, followed by the
  next bytes \texttt{0x6c}=`l', \texttt{0x64}=`d' and
  \texttt{0x00}=`\textbackslash{}0'
\end{itemize}

\begin{enumerate}
\def\labelenumi{\arabic{enumi}.}
\setcounter{enumi}{4}
\item
  \textbf{In the following code, what is going to be printed after
  \texttt{y=}? (note: the answer is not a specific value.) Why does this
  happen?}

\begin{verbatim}
cprintf("x=%d y=%d", 3);
\end{verbatim}
\end{enumerate}

\begin{itemize}
\itemsep1pt\parskip0pt\parsep0pt
\item
  Since cprintf uses va\_arg, ap will initially point to the address of
  the memory location holding 3, and for \texttt{y} ap points to the
  value on the stack, after 3, which will be some junk value.
\end{itemize}

\begin{enumerate}
\def\labelenumi{\arabic{enumi}.}
\setcounter{enumi}{5}
\itemsep1pt\parskip0pt\parsep0pt
\item
  \textbf{Let's say that GCC changed its calling convention so that it
  pushed arguments on the stack in declaration order, so that the last
  argument is pushed last. How would you have to change cprintf or its
  interface so that it would still be possible to pass it a variable
  number of arguments?}
\end{enumerate}

\begin{itemize}
\itemsep1pt\parskip0pt\parsep0pt
\item
  In this case we simply need to reverse the order. We start with the
  end of the list, and we keep decreasing the address instead of
  increasing it.
\end{itemize}

\subsection{Exercise 9}

\begin{itemize}
\itemsep1pt\parskip0pt\parsep0pt
\item
  The kernel initializes its stack at \texttt{line 77} in
  \texttt{kern/entry.S} (\texttt{movl \$(bootstacktop),\%esp}).
\item
  By checking \texttt{obj/kern/kernel.asm}, we see that
  \texttt{bootstacktop} takes the value \texttt{0xf0110000}. Hence, the
  stack is located at this location.
\item
  The kernel reserves space for the stack at \texttt{line 93} in
  \texttt{kern/entry.S} (\texttt{.space KSTKSIZE}). \texttt{KSTKSIZE} is
  defined in \texttt{inc/memlayout.h} as \texttt{(8*PGSIZE)} with
  \texttt{PGSIZE} defined in \texttt{inc/mmu.h} as equal to
  \texttt{4096} bytes. Hence, we get 8*4096 bytes = 32768 bytes = 8192
  stack frames, each having a size of 4 bytes.
\item
  Since the stack grows downwards in an x86 system, the stack pointer is
  initialised to the top of of the stack(\texttt{bootstacktop}), and as
  values are pushed, the value of the stack pointer is reduced by 4.
\end{itemize}

\subsection{Exercise 10}

\begin{itemize}
\itemsep1pt\parskip0pt\parsep0pt
\item
  \texttt{\%ebp} gets pushed onto the stack.
\item
  \texttt{\%ebx} gets pushed onto the stack.
\item
  \texttt{mov \%ebx,0x4(\%esp)} pushes \texttt{\%ebx} onto the stack as
  a parameter (\texttt{x}) to \texttt{cprintf()}.
\item
  \texttt{movl \$0xf0101b60,(\%esp)} pushes the hex value onto the
  stack, which is the string parameter to \texttt{cprintf()}.
\item
  The \texttt{call} to \texttt{cprintf()} pushes the return address
  \texttt{0xf010005a} onto the stack.
\item
  \texttt{mov \%eax,(\%esp)} pushes \texttt{\%eax}, the value of
  \texttt{(x-1)} onto the stack.
\item
  The \texttt{call} to \texttt{test\_backtrace()} pushes the return
  address \texttt{0xf0100069} onto the stack.
\item
  Hence for the \texttt{test\_backtrace()} function, \texttt{\%ebx}, the
  arguments, and \texttt{\%eip} of the return address are pushed before
  the function is called, followed by the pushing in of \texttt{\%ebp}
  when the function starts.
\end{itemize}

\subsection{Exercise 11}

\begin{itemize}
\item
  The following code is added at \texttt{line 63} in
  \texttt{kern/monitor.c}:

\begin{verbatim}
cprintf("Stack backtrace:\n");

int* ebp = (int*)read_ebp();
int eip, i;

while (ebp != 0) {
    eip = *(ebp+1);
    cprintf("ebp %08x  eip %08x  args ", ebp, eip);
    for(i = 1; i <= 5; i ++) {
        cprintf("%08x ", *(ebp+i+1));
    }
    cprintf("\n");
    ebp = (int*)(*ebp);
}
\end{verbatim}
\item
  \texttt{\%ebp} is the current value of \texttt{ebp}.
\item
  The value at the 32-bit location above \texttt{\%ebp} is the value of
  \texttt{eip}. Hence, pointer arithmetic is used to add 4 bytes to
  \texttt{\%ebp} and \texttt{*} is used to access the
  value at the location.
\item
  Since the arguments are the values above \texttt{eip}, pointer
  arithmetic is once again used to add 8-24 bytes to acess the first 5
  arguments.
\end{itemize}

\subsection{Exercise 12}

\begin{itemize}
\item
  The above code at \texttt{line 63} in \texttt{kern/monitor.c} is
  modified into:

\begin{verbatim}
struct Eipdebuginfo info;
cprintf("Stack backtrace:\n");

int* ebp = (int*)read_ebp();
int eip, i;

while (ebp != 0) {
    eip = *(ebp+1);
    cprintf("ebp %08x  eip %08x  args ", ebp, eip);
    for(i = 1; i <= 5; i ++) {
        cprintf("%08x ", *(ebp+i+1));
    }
    cprintf("\n");

    //debug info
    debuginfo_eip(eip, &info);
    cprintf("       %s:%d: %.*s+%d\n", info.eip_file,
    info.eip_line, info.eip_fn_namelen, info.eip_fn_name, eip-info.eip_fn_addr);

    ebp = (int*)(*ebp);
}
\end{verbatim}
\item
  Here we have added code to display additional information using the
  \texttt{debuginfo\_eip()} method.
\item
  In \texttt{debuginfo\_eip()} in \texttt{kern/kdebug.c} the following
  lines of code are added at \texttt{line 183}:

\begin{verbatim}
stab_binsearch(stabs, &lline, &rline, N_SLINE, addr);
if(lline <= rline) {
    info->eip_line = stabs[lline].n_desc;
} else {
    return -1;
}
\end{verbatim}
\item
  This code enables us to get the line number of the line of code to
  which \texttt{eip} points.
\item
  The stab structure and constants are defined in \texttt{inc/stab.h},
  which is from where \texttt{debuginfo\_eip()} gets the
  \texttt{\_\_STAB\_*} structure details from.
\item
  The bootloader loads the stab data at \texttt{0x001022fc}, and links
  to it at \texttt{0xf01022fc} (The value of \texttt{stabs} in
  \texttt{debuginfo\_eip()}), which is the address at which all the
  accesses to the symbol table are made.
\end{itemize}

\end{document}
