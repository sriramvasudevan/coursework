\documentclass[]{article}
\usepackage[T1]{fontenc}
\usepackage{lmodern}
\usepackage{amssymb,amsmath}
\usepackage{ifxetex,ifluatex}
\usepackage{fixltx2e} % provides \textsubscript
% use microtype if available
\IfFileExists{microtype.sty}{\usepackage{microtype}}{}
% use upquote if available, for straight quotes in verbatim environments
\IfFileExists{upquote.sty}{\usepackage{upquote}}{}
\ifnum 0\ifxetex 1\fi\ifluatex 1\fi=0 % if pdftex
  \usepackage[utf8]{inputenc}
\else % if luatex or xelatex
  \usepackage{fontspec}
  \ifxetex
    \usepackage{xltxtra,xunicode}
  \fi
  \defaultfontfeatures{Mapping=tex-text,Scale=MatchLowercase}
  \newcommand{\euro}{€}
\fi
\usepackage[margin=0.8in]{geometry}
\ifxetex
  \usepackage[setpagesize=false, % page size defined by xetex
              unicode=false, % unicode breaks when used with xetex
              xetex]{hyperref}
\else
  \usepackage[unicode=true]{hyperref}
\fi
\hypersetup{breaklinks=true,
            bookmarks=true,
            pdfauthor={Sriram V - CS11B058},
            pdftitle={Lab 3 - Part A : User Environments and Exception Handling},
            colorlinks=true,
            urlcolor=blue,
            linkcolor=magenta,
            pdfborder={0 0 0}}
\urlstyle{same}  % don't use monospace font for urls
\setlength{\parindent}{0pt}
\setlength{\parskip}{6pt plus 2pt minus 1pt}
\setlength{\emergencystretch}{3em}  % prevent overfull lines
\setcounter{secnumdepth}{0}

\title{Lab 3 - Part A : User Environments and Exception Handling}
\author{Sriram V - CS11B058}
\date{}

\begin{document}
\maketitle

\subsection{Exercise 1}

\begin{itemize}
\itemsep1pt\parskip0pt\parsep0pt
\item
  Modified \texttt{mem\_init()} in \texttt{kern/pmap.c} to allocate and
  map the \texttt{envs} array.
\item
  The added code is very similar to the allocation and mapping of
  \texttt{pages} in the previous lab. This is because although the
  previous lab dealt with the management of pages and the current one
  with user environments, both need to be kept track of by the kernel,
  and this book-keeping for both is done by different \texttt{struct}s.
\item
  \texttt{NENV} instances (pointed to by \texttt{envs}) of the
  \texttt{Env} structure are initialised with the help of
  \texttt{boot\_alloc()} at \texttt{line 162}.
\item
  \texttt{envs} is also mapped to \texttt{UENVS} (read-only), which
  allows user processes to access this data without modifying it. This
  is done by means of the \texttt{boot\_map\_region()} function, which
  allocates the necessary pages and adds entries in the Page Directory.
  (\texttt{line 197} in \texttt{kern/pmap.c})
\item
  The kernel however has read-write access because it accesses
  \texttt{envs} at a higher virtual address that has RW permissions for
  the kernel, but cannot be accessed by the user. (Done by the mapping
  at \texttt{line 222}).
\item
  Running \texttt{make qemu} after this gives a
  \texttt{check\_kern\_pgdir()} success message.
\end{itemize}

\subsection{Exercise 2}

\begin{itemize}
\itemsep1pt\parskip0pt\parsep0pt
\item
  Now that the code to create \texttt{envs} is done, we fill in the code
  to actually create and run a user environment. Code to update
  \texttt{envs} (to help the kernel keep track of user environments) is
  also added in. All this is written in the \texttt{kern/env.c} file.
\item
  The \texttt{cprintf} verb \texttt{\%e} is used in the \texttt{panic()}
  calls to print out descriptions corresponding to error codes.
\end{itemize}

\begin{enumerate}
\def\labelenumi{\arabic{enumi}.}
\itemsep1pt\parskip0pt\parsep0pt
\item
  \textbf{\texttt{env\_init()}:}

  \begin{itemize}
  \itemsep1pt\parskip0pt\parsep0pt
  \item
    This function initialises the \texttt{Env} structures in the
    \texttt{envs} array and adds it to the \texttt{env\_free\_list}
    linked list. (\texttt{lines 114 - 134})
  \item
    \texttt{env\_free\_list} is initially set to \texttt{NULL}.
  \item
    Since the environments in the free list are to be in the same order
    as they are in the \texttt{envs} array, \texttt{envs} is looped over
    in reverse order, so that when an element of \texttt{envs} is being
    added to the list, its \texttt{env\_link} can be made to point to
    the element that is next in the free list array.
  \item
    The status is set to \texttt{ENV\_FREE} for each of the elements,
    and their runs are initialised to \texttt{0}.
  \item
    Finally \texttt{env\_init\_percpu()} is called, which configures the
    segmentation hardware with separate segments for the kernel(PL 0)
    and user(PL 3).
  \end{itemize}
\item
  \textbf{\texttt{env\_setup\_vm()}:}

  \begin{itemize}
  \itemsep1pt\parskip0pt\parsep0pt
  \item
    In this function, the kernel virtual memory layout for an
    environment is initialised. (\texttt{lines 167 - 206})
  \item
    A page directory is allocated using \texttt{page\_alloc()}
    (initialised with all bytes equal to \texttt{\textbackslash{}0}).
  \item
    The kernel's page directory is copied into the environment's page
    directory. This is because the virtual address space of all
    environments is identical above \texttt{UTOP}, except at
    \texttt{UVPT}. This virual axdress space contains the kernel. Hence,
    the kernel's virtual address mappings are copied in.
  \item
    The reference counter for the page directory is incremented as well,
    since it is inserted into itself, at \texttt{UVPT}.
  \end{itemize}
\item
  \textbf{\texttt{region\_alloc()}:}

  \begin{itemize}
  \itemsep1pt\parskip0pt\parsep0pt
  \item
    This function allocates \texttt{len} bytes of physical memory for an
    environment, and also mapes it at a virtual address that is in the
    environment's address space. (\texttt{lines 278 - 297})
  \item
    As required by the implementation of the function, the passed
    \texttt{va} and \texttt{va+len} are rounded down and rounded up
    respectively, so that the usage of this function becomes simpler.
  \item
    Physical memory for this environment is then allocated in the form
    of pages (using \texttt{page\_alloc()}) and inserting them into the
    \texttt{env\_pgdir} (using \texttt{page\_insert()}) with the
    relevant permissions (user and kernel RW).
  \end{itemize}
\item
  \textbf{\texttt{load\_icode()}:}

  \begin{itemize}
  \itemsep1pt\parskip0pt\parsep0pt
  \item
    Since we don't have a filesystem as yet, the kernel is set up to
    load static binary images that are embedded within the kernel
    itself. These binaries are in the \texttt{ELF} format.
  \item
    The \texttt{load\_icode()} function loads a passed binary into the
    passed environment, and sets up its stack and processor flags
  \item
    A good place for reference is the \texttt{boot/main.c} file, since
    the loading of a binary into an environment is very similar to the
    loading of the kernel into the memory by the bootloader.
  \item
    The binary is first tested to be of a valid \texttt{ELF} format,
    with the help of the \texttt{ELF\_MAGIC} value.
  \item
    \texttt{ph} is then set to point to the start of the program header
    structures in the binary.
  \item
    The page directory is also set to the passed environment's (using
    \texttt{lcr3()}) page directory (even though the environment is not
    running, but is simply being loaded into), so that the virtual
    address mappings for this environment are set up, thus making the
    loading of the program segments at the relevant virtual addresses
    much simpler.
  \item
    For each loadable program segment (specified by a \texttt{p\_type}
    of \texttt{ELF\_PROG\_LOAD}), a region of physical memory is
    allocated, and the binary is moved into this physical space using
    \texttt{memmove()}.
  \item
    The remaining memory is set to \texttt{0} using \texttt{memset()}.
  \item
    The file size must be less than the memory allocated for a
    particular loadable program segment, otherwise an error must be
    thrown.
  \item
    The environment's \texttt{EIP} is set to the entry point for the
    code, mentioned in the binary's \texttt{e\_entry} variable. The EIP
    is stored in the Trapframe associated with an environment.
  \item
    Finally, physical memory for the stack (for the loaded binary) is
    allocated with \texttt{region\_alloc()}, of size \texttt{PGSIZE}.
  \item
    The page directory to be used is also forced back to the kernel's
    page directory, using the \texttt{lcr3()} function.
  \end{itemize}
\item
  \textbf{\texttt{env\_create()}:}

  \begin{itemize}
  \itemsep1pt\parskip0pt\parsep0pt
  \item
    Used to create a new environment, load in a passed binary, and set
    the environment type to the passed type.
  \item
    The new environment is created using the \texttt{env\_alloc()}
    function. The parent ID of the new environment is set to \texttt{0}.
  \item
    Failure to allocate a new environment results in a panic.
  \item
    Otherwise, the binary is loaded into the newly created environment
    using \texttt{load\_icode()}
  \item
    The \texttt{env\_type} variable for the environment is set to the
    passed variable, \texttt{type}.
  \end{itemize}
\item
  \textbf{\texttt{env\_run()}:}

  \begin{itemize}
  \itemsep1pt\parskip0pt\parsep0pt
  \item
    This function enables a context switch from the current environment
    to the passed environment.
  \item
    If a current environment (\texttt{curenv}) exists (not the first
    call to this function), then set its status from
    \texttt{ENV\_RUNNING} to \texttt{ENV\_RUNNABLE}.
  \item
    \texttt{curenv} is set to the passed environment, and its status is
    set to \texttt{ENV\_RUNNING}.
  \item
    The number of runs for this environment is incremented, and
    \texttt{lcr3()} is used to switch to its address space, by forcing
    the use of its page directory.
  \item
    A call to \texttt{env\_pop\_tf()} is made with the passed
    environment's Trapframe as an argument.
  \item
    This restores the register values in the Trapframe with the
    \texttt{iret} instruction, exits the kernel, and starts executing
    the environment's code.
  \end{itemize}
\end{enumerate}

\subsection{Exercise 3}

\begin{itemize}
\itemsep1pt\parskip0pt\parsep0pt
\item
  Read the chapter on `Exceptions and Interrupts' in the 80386
  Programmer's Manual and the IA-32 Developer's Manual, and understood
  the differences between exceptions and interrupts with regard to
  Intel.
\end{itemize}

\subsection{Exercise 4}

\begin{itemize}
\itemsep1pt\parskip0pt\parsep0pt
\item
  When an interrupt/exception occurs, control is transferred to the
  kernel.
\item
  However, in order to ensure that the protected control transfers are
  actually \emph{protected}, the processor's interrupt/exception
  mechanism is designed so that the code running currently when the
  interrupt or exception occurs doesn't get to choose arbitrarily where
  and how the kernel is to be entered.
\item
  Hence, we need to specify entry points into the kernel for the various
  exceptions. In this Exercise, we only handle the processor exceptions.
\item
  For the 20 processor defined exceptions (given by \texttt{\#define}s
  in \texttt{inc/trap.h}), we add an entry point by using either the
  \texttt{TRAPHANDLER()} or \texttt{TRAPHANDLER\_NOEC()} macro,
  depending on whether the exception has an associated error code or not
  respectively. Whether an exception pushes an error code on the stack
  can be found out by looking at the Error Code Summary in the 80386
  Programmer's Manual.
\item
  For a particular exception, we can specify any name that we wish as
  the first parameter of the appropriate macro, and this enables us to
  pass the exception handling routine as a function pointer (during IDT
  setup), by using this name.
\item
  Certain exceptions are reserved by Intel (9 and 15 in the first 20
  exceptions), and since they will never be generated by the processor
  we can handle them in any way as we desire.
\item
  The macro creates the exception handler routine for an excepion. The
  exception handler routine pushes a trap number and either an error
  code or a zero onto the stack, and jumps to \texttt{\_alltraps}.
\item
  The code for \texttt{\_alltraps} in \texttt{kern/trapentry.S} pushes
  the DS and ES segment descriptors onto the stack, which is followed by
  pushing in all the 8 registers. It is done in this order because it
  fits in nicely with the layout of \texttt{strcut Trapframe}.
  (Trapframe registers on the top, followed by \texttt{ES}, \texttt{DS},
  the trap number, error code etc.)
\item
  Now the kernel's data segment descriptor (at \texttt{GD\_KD}) is
  loaded into \texttt{DS} and \texttt{ES}.
\item
  \texttt{ESP} is pushed onto the stack, to pass the built up Trapframe
  as an argument to \texttt{trap}. (Passing \texttt{ESP} means that a
  pointer to the Trapframe is being passed.)
\item
  A \texttt{call} to \texttt{trap} is made.
\item
  It doesn't look like \texttt{trap} can return, since a dispatch is
  made depending on what type of trap occured, which is then followed by
  a call to \texttt{env\_run()}, which we know exits the kernel and
  enters the user environment.
\end{itemize}

\subsubsection{Challenge:}

\begin{itemize}
\item
  I got rid of the similar looking code in \texttt{kern/trapentry.S} and
  \texttt{kern/trap.c}, by making use of an array, and adding the
  function names to this array as and when they are created.
\item
  I referred to the xv6 source code, which makes use of a
  \texttt{vectors} array for this purpose.
\item
  In \texttt{trapentry.S}, I defined the array with the code:

\begin{verbatim}
.data
.globl thandlers
thandlers:
\end{verbatim}
\item
  I also modified the two macros by adding a \texttt{.text} in the
  beginning (to specify that the following code is part of the program's
  text portion) and added

\begin{verbatim}
.data;
.long name
\end{verbatim}

  at the end (to specify that this bit of code is part of the program's
  data section).
\item
  Thus, this builds an array of function pointers to interrupt handler
  routines.
\item
  By specifying \texttt{extern int thandlers{[}{]};} at \texttt{line 61}
  in \texttt{kern/trap.c}, we can use \texttt{thandlers{[}i{]}} in a
  \texttt{for} loop (that has \texttt{SETGATE()} as the body), instead
  of having multiple lines of \texttt{SETGATE()} calls.
\item
  Also, the one \texttt{extern} array declaration helps us to get rid of
  the multiple \texttt{extern void NAME();} declarations that were
  required earlier.
\end{itemize}

\subsubsection{Questions}

\begin{enumerate}
\def\labelenumi{\arabic{enumi}.}
\itemsep1pt\parskip0pt\parsep0pt
\item
  \textbf{What is the purpose of having an individual handler function
  for each exception/interrupt? (i.e., if all exceptions/interrupts were
  delivered to the same handler, what feature that exists in the current
  implementation could not be provided?)}
\end{enumerate}

\begin{itemize}
\itemsep1pt\parskip0pt\parsep0pt
\item
  The purpose of having an individual handler function for each
  exception/interrupt is so that the OS can have different values on the
  stack for different interrupts/exceptions.
\item
  For those that take an error code, their error code may be pushed onto
  the stack. For those that don't, a zero is pushed. Apart from this,
  different values may also be pushed onto the stack specifically for
  different interrupts/exceptions.
\item
  If all exceptions/interrupts were delivered to the same handler, then
  distinct trap numbers wouldn't be pushed onto the stack before calling
  \texttt{trap}. Thus, the \texttt{trap()} function wouldn't be able to
  distinguish between different exceptions/interrupts. The hardware too
  is incapable of distinguishing between trap errors.
\item
  Without this distinguishing feature present in the kernel code, if we
  wish to handle different interrupts/exceptions, we would have to allow
  the user to invoke the exception/interrupt, and this is a serious
  security issue.
\end{itemize}

\begin{enumerate}
\def\labelenumi{\arabic{enumi}.}
\setcounter{enumi}{1}
\itemsep1pt\parskip0pt\parsep0pt
\item
  \textbf{Did you have to do anything to make the user/softint program
  behave correctly? The grade script expects it to produce a general
  protection fault (\texttt{trap 13}), but \texttt{softint}'s code says
  \texttt{int \$14}. Why should this produce interrupt vector 13? What
  happens if the kernel actually allows \texttt{softint}'s
  \texttt{int \$14} instruction to invoke the kernel's page fault
  handler (which is interrupt vector 14)?}
\end{enumerate}

\begin{itemize}
\itemsep1pt\parskip0pt\parsep0pt
\item
  No, I didn't have to do anything to make it behave correctly. The
  \texttt{softint} program tries to invoke the page fault handler on its
  own, but we shouldn't allow users to invoke exceptions of their
  choice. This is ensured by setting \texttt{DPL=0} while filling up the
  IDT using \texttt{SETGATE()}(the last parameter) -- \texttt{line 72}
  in \texttt{trap/entry.c}.
\item
  Thus, when the user at a numerically higher privilege level
  (\texttt{PL = 3}) tries to access code at a numerically lower
  privilege level (\texttt{PL = 0}), a General Protection Fault
  (\texttt{trap 13}) is triggered instead, to handle the error of a
  piece of code violating privilege rules.
\item
  If we did allow the user to invoke exceptions as and when they choose,
  it can result in the execution of malicious code with kernel
  privileges. The user can push in malicious code into the stack, and
  then make a call to an exception, which would pick up the malicious
  code off the stack and execute it with kernel privileges.
\end{itemize}

\end{document}
